\documentclass{article}

% Packages
\usepackage{graphicx} % For including images
\usepackage{amsmath} % For mathematical symbols and equations
\usepackage{hyperref} % For hyperlinks
\usepackage{listings} % For including code snippets

% Title
\title{Assignment 3 Report}
\author{Tom Gerson, Sebastien Boo}
\date{March 8, 2024}

\begin{document}
	
	\maketitle
	
	
	\section*{Task 1}
		\paragraph{a)}
		Adding padding to the image : 
		\[
		\begin{matrix}
			0 & 0 & 0 & 0 & 0 & 0 & 0 \\
			0 & 2 & 1 & 2 & 3 & 1 & 0\\
			0 & 3 & 9 & 1 & 1 & 4 & 0\\
			0 & 4 & 5 & 0 & 7 & 0 & 0\\
			0 & 0 & 0 & 0 & 0 & 0 & 0\\
		\end{matrix}
		\]
		We then apply hadamard product between the kernel matrix and each of the matrix below, obtained by sliding the kernel with stride = 1 :
		\[
		\begin{bmatrix}
			0 & 0 & 0 \\
			0 & 2 & 1 \\
			0 & 3 & 9 \\
		\end{bmatrix}
		\quad
		\begin{bmatrix}
			0 & 0 & 0 \\
			2 & 1 & 2 \\
			3 & 9 & 1 \\
		\end{bmatrix}
		\quad
		\begin{bmatrix}
			0 & 0 & 0 \\
			1 & 2 & 3 \\
			9 & 1 & 1 \\
		\end{bmatrix}
		\quad
		\begin{bmatrix}
			0 & 0 & 0 \\
			2 & 3 & 1 \\
			1 & 1 & 4 \\
		\end{bmatrix}
		\quad
		\begin{bmatrix}
			0 & 0 & 0 \\
			3 & 1 & 0 \\
			1 & 4 & 0 \\
		\end{bmatrix}
		\]
		\[
		\begin{bmatrix}
			0 & 2 & 1 \\
			0 & 3 & 9 \\
			0 & 4 & 5 \\
		\end{bmatrix}
		\quad
		\begin{bmatrix}
			2 & 1 & 2 \\
			3 & 9 & 1 \\
			4 & 5 & 0
		\end{bmatrix}
		\quad
		\begin{bmatrix}
			1 & 2 & 3\\
			9 & 1 & 1\\
			5 & 0 & 7
		\end{bmatrix}
		\quad
		\begin{bmatrix}
			2 & 3 & 1\\
			1 & 1 & 4\\
			0 & 7 & 0
		\end{bmatrix}
		\quad
		\begin{bmatrix}
			3 & 1 & 0\\
			1 & 4 & 0\\
			7 & 0 & 0
		\end{bmatrix}
		\]
		\[
		\begin{bmatrix}
			0 & 3 & 9 \\
			0 & 4 & 5 \\
			0 & 0 & 0 \\
		\end{bmatrix}
		\quad
		\begin{bmatrix}
			3 & 9 & 1 \\
			4 & 5 & 0 \\
			0 & 0 & 0 \\
		\end{bmatrix}
		\quad
		\begin{bmatrix}
			9 & 1 & 1 \\
			5 & 0 & 7 \\
			0 & 0 & 0 \\
		\end{bmatrix}
		\quad
		\begin{bmatrix}
			1 & 1 & 4 \\
			0 & 7 & 0 \\
			0 & 0 & 0 \\
		\end{bmatrix}
		\quad
		\begin{bmatrix}
			1 & 4 & 0 \\
			7 & 0 & 0 \\
			0 & 0 & 0 \\
		\end{bmatrix}
		\]
		
	We obtain : 
	
		\[
		\begin{bmatrix}
			0 & 0 & 0 \\
			0 & 0 & 2 \\
			0 & 0 & 9 \\
		\end{bmatrix}
		\quad
		\begin{bmatrix}
			 0 & 0 & 0 \\
			-4 & 0 & 4 \\
			-3 & 0 & 1 \\
		\end{bmatrix}
		\quad
		\begin{bmatrix}
			 0 & 0 & 0 \\
			-2 & 0 & 6 \\
			-9 & 0 & 1 \\
		\end{bmatrix}
		\quad
		\begin{bmatrix}
			 0 & 0 & 0 \\
			-4 & 0 & 2 \\
			-1 & 0 & 4 \\
		\end{bmatrix}
		\quad
		\begin{bmatrix}
		 	 0 & 0 & 0 \\
			-6 & 0 & 0 \\
			-1 & 0 & 0 \\
		\end{bmatrix}
		\]
	
		\[
		\begin{bmatrix}
			0 & 0 & 1 \\
			0 & 0 & 18 \\
			0 & 0 & 5 \\
		\end{bmatrix}
		\quad
		\begin{bmatrix}
			-2 & 0 & 2 \\
			-6 & 0 & 2 \\
			-4 & 0 & 0
		\end{bmatrix}
		\quad
		\begin{bmatrix}
			 -1 & 0 & 3\\
			-18 & 0 & 2\\
		 	 -5 & 0 & 7
		\end{bmatrix}
		\quad
		\begin{bmatrix}
			-2 & 0 & 1\\
			-2 & 0 & 8\\
			 0 & 0 & 0
		\end{bmatrix}
		\quad
		\begin{bmatrix}
			-3 & 0 & 0\\
			-2 & 0 & 0\\
			-7 & 0 & 0
		\end{bmatrix}
		\]
	
		\[
		\begin{bmatrix}
			0 & 0 & 9 \\
			0 & 0 & 10 \\
			0 & 0 & 0 \\
		\end{bmatrix}
		\quad
		\begin{bmatrix}
			-3 & 0 & 1 \\
			-8 & 0 & 0 \\
			 0 & 0 & 0 \\
		\end{bmatrix}
		\quad
		\begin{bmatrix}
			 -9 & 0 & 1 \\
			-10 & 0 & 14 \\
			  0 & 0 & 0 \\
		\end{bmatrix}
		\quad
		\begin{bmatrix}
			-1 & 0 & 4 \\
			 0 & 0 & 0 \\
			 0 & 0 & 0 \\
		\end{bmatrix}
		\quad
		\begin{bmatrix}
			 -1 & 0 & 0 \\
			-14 & 0 & 0 \\
			  0 & 0 & 0 \\
		\end{bmatrix}
		\]
	
	
	
		\paragraph{b)}
	
	\section*{Task 2}

	
\end{document}
